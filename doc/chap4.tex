\chapter{Conclusion}

\section{Difficultés rencontrées}
    Nous avons rencontré quelques difficultés lors de la réalisation de ce projet, en voici quelques-uns.
    \medbreak
    \begin{itemize}
        \item[$\bullet$] Lors des premiers affichages il nous était impossible d'afficher le monde en raison d'un problème d'initialisation des structures. Nous passions par une variable globale ce qui empêchait de déclarer plusieurs variables de chaque structure.
        \medbreak
        \item[$\bullet$] Nous avons également remarqué des problèmes en lien avec la taille de notre monde, en effet, pour des plateaux très très grands (de l'ordre de 10000 cases, et avec moitié moins de pions), notre jeu ne fonctionne plus pour ce qui nous semble être des raisons de gestion de mémoire. Ce problème est querellé avec le nombre maximum de joueurs. Nous pensons que ce problème pourrait être résolu en utilisant des \texttt{mallocs()} et des \texttt{free()}. 
    \end{itemize}

\section{Bilan du Projet}
    Nous avons réussi à faire un jeu qui fonctionne de 2 à 10 joueurs (voir plus en fonction de la taille du monde), sur un plateau torique de 9 à 25 000 cases, avec 4 types de pièces différentes.\\
    Le jeu admet 2 possibilités de départs et de victoires. Il comprend aussi des changements de terrain en cours de partie (avec 3 géométries disponibles) et un système de capture et de libération de pièce. La partie peut se jouer toute seule en choisisant aléatoirement les pièces et leurs déplacements. Le plateau est affiché tour par tour et affiche aussi à l'utilisateur les coups joués et des informations sur les événements du jeu.
    \medbreak
    \noindent Il nous resterait maintenant à implémenter des déplacements guidés pour permettre aux pièces de se déplacer en direction des objectifs de fin de partie.
    
\section{Ce que le projet nous a apporté}
    Au cours de ces dernières semaines, nous avons appris à travailler ensemble sur un même code et à se corriger mutuellement. Ainsi, nous nous sommes amélioré en \texttt{C} et avons compris certaines suptilitées du language qui nous étaient encore inconnues. Le projet nous a forgé à l'utilisation d'outils comme \texttt{Git}, indispensable pour travailler en groupe de manière efficace, ou encore \texttt{Makefile}, qui facilite grandement les tâches répétitives des projets de développements. Ces compétences nous seront indispensables dans la suite de notre parcours.  \\
    Ce projet nous a également appris à produire un rendu qui respecte des demandes précises tout en ayant un code lisible et compréhensible. \\
    Pour finir, l'autonomie laissée lors de ce projet nous aussi a permis de répondre au sujet de manière assez libre. Ceci nous a fait réfléchir à la manière de lier les différentes structures et données du projet pour créer par nous-mêmes notre architecture de jeu.