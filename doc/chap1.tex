\chapter{Introduction}
Notre projet se base sur l'exemple d'un \textit{mansuba}, considéré comme l'ancêtre Perse des échecs dont le principe majeur est de mettre l'adversaire mat. Le but est donc d'implémenter tout un système de jeu et de relation afin de produire un jeu de plateau dans l'esprit du mansuba ou des échecs.


\section{Présentation du projet}
    \subsection{Règles}
        Le jeu est soumis à certaines règles afin d'encadrer sa création. Pour mieux comprendre les différentes implémentations, voici les principales :
        \medbreak
        \begin{itemize}
            \item[\textdagger] Le monde: Il est représenté par un ensemble de positions. Il est également définit par sa longueur et sa largeur qui en les multipliants nous donnent le nombre maximum de positions.
            \medbreak
            \item[\textdagger] Les relations : C'est ce qui relie les positions entre elles en fonction des directions. On peut de ce fait définir \textbf{les voisins} comme étant toutes les positions qui sont reliées entre elles. On a donc un maximum de voisins en fonction des déplacements autorisés.
            \medbreak
            \item[\textdagger] Déplacement simple: Il est définit comme étant le changement de position pour une position voisine disponible.
            \medbreak
            \item[\textdagger] Saut simple : Il est possible, si la position voisine n'est pas disponible, de sauter par dessus l'obstacle. Cependant si la position derrière est occupée ou que la structure du monde ne le permet pas, le saut est impossible.
            \medbreak
            \item[\textdagger] Victoire : Elle définit l'arret du jeu, soit car le nombre de tour maximum est atteint, soit car un joueur a réussi à atteindre les positions de départ de l'autre joueur avec une de ces pièces.
        \end{itemize}
        
    \subsection{Stratégie de résolution}
        Notre stratégie pour effectuer ce projet est basée sur les tests ainsi que sur la modularité.
        En effet tout au long du projet nous avons fait  en sorte que toutes les valeurs utilisées puissent varier en fonctions des demandes de l'utilisateur. On a donc utilisé le moins de constantes possible pour pouvoir répondre à des changements basiques sans rajouter ou changer tout une partie du projet. \\
        Egalement, notre méthode pour implémenter de nouvelles fonctionalités est dite par les tests. Pour cela on fait une liste d'objectifs de la fonctionnalité, chaque objectif est testé avant de passer au suivant pour être sûr que la fonctionnalité marche correctement une fois implémentée.